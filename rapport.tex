\documentclass{article}
\usepackage[utf8]{inputenc}
\usepackage[french]{babel}
\usepackage[]{amsmath}
\usepackage{listings}
\usepackage{graphicx}
\usepackage{geometry}
\usepackage[]{algorithm2e}
\geometry{hmargin=2.5cm}


\title{Projet Graphs and Networks\\Recherche de répétitions dans les séquences génomiques}
\author{Antoine Marendet}
\date{26 novembre 2014}

\begin{document}

\maketitle

\section{Objectif}

L'objectif de ce projet est de répérer des répétitions de sous-séquences dans une séquence d'ADN $S$ en répérant des structures particulières dans un graphe, dit \textit{de Brujin}, associé à la séquence $S$.

\section{Utilisation du programme}

Le programme est compilé en utilisant \textit{javac} en lançant le script \textit{compile.sh}.\\
Le programme est lancé en appelant le script \textit{run.sh} donc l'usage est le suivant :
\begin{center}
	./run.sh <fichier de données>
\end{center}
Les fichiers de donnée contiennent sur la première ligne le nombre $k$ et sur les lignes suivantes la séquence d'ADN.
Les résultats sont affichés dans le terminal et écrits dans le dossier \textit{res}.

\section{Modélisation}

\subsection{Définitions}
Un k-mer est une sous-séquence de longueur $k$ de $S$.\\
Le graphe de Brujin $G(S,k)$ associé à $S$ pour un entier $k$ est défini ainsi : 
\begin{itemize}
\item les sommets sont tous les k-mer de $S$,
\item il existe un arc du k-mer $x$ vers le k-mer $y$ si et seulement si :
\begin{itemize}
\item $x[2..k] = y[1..k-1]$,
\item $x[1..k].y[k]$ est un (k+1)-mer de $S$.
\end{itemize}
\end{itemize} 

Dans le programme, le graphe est construit de cette façon :
\begin{enumerate}
\item pour toutes les sous-chaînes $x$ de longueur $k$ de $S$ :
    \begin{enumerate}
    \item si 
    \end{enumerate}
\end{enumerate}

\subsection{Répétitions en tandom}
Dans un premier temps, on cherche un moyen de repérer les répétitions en tandem dans la séquence.\\
Une répétition en tandem $RT(d, r, m)$ est caractérisée par sa position de départ $d$, son nombre de répétitions $r$, et le motif $m$.
$RT(d,rm)$ est une répétition en tandem si et seulement si :
\[
\exists d,r \in \Re, r \ge 2, \forall i \in {1..r}, S[d+k*i..d+k*(i+1)] = m
\]
De telles répétitions se traduisent par des circuits élémentaires de longueur $k$ dans le graphe de Brujin. Malheureusement, nous n'avons pas trouvé de manière, à partir de la connaissance des ces circuits d'en déduire directement le k-mer qui présente une répétition en tandem et le nombre de ces répétitions.\\
En pratique, on va donc commencer par rechercher les circuits élémentaires de longueur $k$ présents dans le graphe, puis pour chaque k-mer contenu dans ces graphes, on va recherche dans la séquence si ces k-mer présentent effectivement des répétitions en tandem, sachant que pour chacun de ces circuits particuliers, on est certains de trouver au moins une répétition en tandem.

\section{Algorithme de résolution}

On commence par lire les données du problème dans le fichier.

\subsection{Construction du graphe de Brujin}

Notre graphe de Brujin est représenté en mémoire par des listes d'adjacence.
Le graphe est construit en parcourant toute la séquence, en groupant les caractères par groupes de $k$. On conserve en mémoire le k-mer précédemment observé. Le nouveau k-mer, qui suit immédiatement le précédent, est un successeur de ce k-mer. On l'ajoute donc à la liste des successeurs de ce k-mer.\\



\begin{algorithm}[H]
	\KwData{Séquence d'ADN $S$, taille des sous-séquences $k$}
	precedent $\leftarrow S[1..k]$\;
	\For{i dans 2..longueur de $S$ - k}{
	    \If{$S[i..i+k]$ n'est pas un successeur de précédent}{
		    ajouter le k-mer $S[i..i+k]$ comme successeur de précédent\;
		}
		précédent $\leftarrow S[i..i+k]$\;
	}
	\caption{Construction du graphe de Brujin en pseudocode}

\end{algorithm}

Cet algorithme a une complexité en $O(n)$, et une implémentation en $O(n)$ également, grâce à l'utilisation de HashSet et HashMap pour le stockage des données, dont la plupart des opérations se font en temps constant.

\subsection{Recherche des cycles élémentaires}

On utilise pour rechercher les cycles élémentaires l'algorithme de Tarjan, tel que fourni dans l'un des papiers proposés.\\

\begin{algorithm}[H]
    \SetKwProg{Fn}{procédure}{}{fin procédure}
    \SetKwProg{For}{pour}{ faire}{fin pour}
    \SetKwProg{If}{si}{ alors}{}
    \SetKwProg{ElseIf}{sinon si}{ alors}{fin si}
    \SetKwProg{uElseIf}{sinon si}{ alors}{}
    \KwData{pStack : pile d'entiers, mStack : pile d'entiers, mark : tableau de booléens, s : entier}
    \Fn{backtrack(entier $v$), retourne un booléen}{
        $f \leftarrow$ faux\;
        mettre $v$ sur pStack\;
        mark$[v] \leftarrow $ vrai\;
        mettre $v$ sur mStack\;
        \For{tous les successeurs $w$ de $v$}{
            \If{$w < s$}{
                supprimer $w$ des successeurs de $v$\;
            }
            \uElseIf{$w = s$}{
                enregistrer le circuit donné par les points contenus dans pStack\;
                $f \leftarrow $ vrai\;
            }
            \ElseIf{$mark[v] = $ faux}{
                $g \leftarrow $ backtrack($w$)\;
                $f \leftarrow f $ ou $g$\;
            }
            
            \If{$f = $ vrai}{
                vider les points $i$ de mStack jusqu'à $v$ inclus et faire $mark[i] \leftarrow $ faux\;
            }
        }
        supprimer $v$ de pStack\;
    }
    
    entier $n \leftarrow $ nombre de noeuds du graphe\;
    \For{pour chaque noeud $i$}{mark$[i] \leftarrow $ faux\;}
    \For{$s$ de $1$ à $n$}{
        backtrack(s)
        enlever les points $i$ de mStack et faire $mark[i] \leftarrow $ faux\;
    }
    \caption{Algorithme de Tarjan pour la recherche de cycles élémentaires}
\end{algorithm}

\end{document}
