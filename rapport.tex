\documentclass{article}
\usepackage[utf8]{inputenc}
\usepackage[french]{babel}
\usepackage[]{amsmath}
\usepackage{listings}
\usepackage{graphicx}
\usepackage{geometry}
\geometry{hmargin=2.5cm}

\title{Projet Graphs and Networks\\Recherche de répétitions dans les séquences génomiques}
\author{Antoine Marendet}
\date{26 novembre 2014}

\begin{document}

\maketitle

\section{Objectif}

L'objectif de ce projet est de répérer des répétitions de sous-séquences dans une séquence d'ADN $S$ en répérant des structures particulières dans un graphe, dit \textit{de Brujin}, associé à la séquence $S$.

\section{Graphe}

Un k-mer est une sous-séquence de longueur $k$ de $S$.\\
Le graphe de Brujin $G(S,k)$ associé à $S$ pour un entier $k$ est défini ainsi : 
\begin{itemize}
\item les sommets sont tous les k-mer de $S$,
\item il existe un arc du k-mer $x$ vers le k-mer $y$ si et seulement si :
\begin{itemize}
\item $x[2..k] = y[1..k-1]$,
\item $x[1..k].y[k]$ est un (k+1)-mer de $S$.
\end{itemize}
\end{itemize} 

Dans le programme, le graphe est construit de cette façon :
\begin{enumerate}
\item pour toutes les sous-chaînes $x$ de longueur $k$ de $S$ :
    \begin{enumerate}
    \item si 
    \end{enumerate}
\end{enumerate}

\section{Répétitions en tandom}
Dans un premier temps, on cherche un moyen de repérer les répétitions en tandem dans la séquence.\\
Une répétition en tandem $RT(d, r, m)$ est caractérisée par sa position de départ $d$, son nombre de répétitions $r$, et le motif $m$.
$RT(d,rm)$ est une répétition en tandem si et seulement si :
\[
\exists d,r \in \Re, r \ge 2, \forall i \in {1..r}, S[d+k*i..d+k*(i+1)] = m
\]
De telles répétitions se traduisent par des circuits élémentaires de longueur $k$ dans le graphe de Brujin. Malheureusement, nous n'avons pas trouvé de manière, à partir de la connaissance des ces circuits d'en déduire directement le k-mer qui présente une répétition en tandem et le nombre de ces répétitions.\\
En pratique, on va donc commencer par rechercher les circuits élémentaires de longueur $k$ présents dans le graphe, puis pour chaque k-mer contenu dans ces graphes, on va recherche dans la séquence si ces k-mer présentent effectivement des répétitions en tandem, sachant que pour chacun de ces circuits particuliers, on est certains de trouver au moins une répétition en tandem.

\end{document}
